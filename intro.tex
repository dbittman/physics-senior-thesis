\section{Introduction}

Boundary value problems, problems defined by a partial differential equation and a set of
constraints on the solution, are found in many areas of physics. These include problems
such as finding the temperature of a rod that has a given temperature on either end,
determining the electric potential inside a region given the potential around the
boundary, or determining the flow of an incompressible fluid. As the complexity of 
these problems increases, however, the feasibility of solving them exactly by hand
quickly decreases. Instead, we have the option to solve them with iterative methods
using computers. These types of problems are well-suited for computers because
they are easily parallelizable~\cite{parallel}. I have developed an
implementation of such a solver for electrostatics and fluid flow utilizing the 
parallelizability of the problem and designed it to be efficient, taking advantage 
of the design of modern CPUs.

Although many differential equations are typically used in boundary value problems,
I will limit the focus to the Poisson equation. Named after Siméon Denis Poisson, the Poisson equation
is used in a number of typical boundary value problems, including electrostatics.
In 2-dimensional Euclidean space it is\cite{boas},
$$\nabla^2 \phi = f,$$
where $f$ is a real-valued function inside the space and $\phi$ is the steady-state
solution for the potential in the space. However, since this solution will be calcuated
numerically on a computer, we must discretize the equation so that it can be used to
represent a solution on a 2-dimensional grid.

\subsection{Discreetizing Poisson's Equation}

\begin{figure}[htb]
	\centering
	\includesvg{grid}
	\caption{Visualization of grid, with indicies.}
\label{grid}
\end{figure}

Consider a grid of size $n$ by $n$ as in figure~\ref{grid}\footnote{In general, it is not difficult to allow a rectangular grid, but for simplicity
we will require the grid to be square.}, with values represented by $u_{i,j}$. We want to apply the Poisson equation to
this $u$ as follows:
$$\nabla^2 u = \frac{\partial^2 u}{\partial x^2} + \frac{\partial^2 u}{\partial x^2} = f,$$
however, $u$ is not continuous, so the partial derivative for $x$ becomes
$$\frac{\partial^2 u}{\partial x^2} \approx \frac{u_{i-1,j} - 2 u_{i,j} + u_{i+1,j}}{h^2},$$
at a grid point $(i,j)$, where $h = 1 / n$ is the width and height of a grid cell. The partial derivative for $y$ is similar, using $j$ instead of $i$.
This gives the full discrete Poisson equation as\cite{poisson-relax}\cite{myths},
$$\frac{u_{i-1,j} - 2 u_{i,j} + u_{i+1,j}}{h^2} + \frac{u_{i,j-1} - 2 u_{i,j} + u_{i,j+1}}{h^2} = f_{i,j}.$$

Since $u_{i,j}$ is what we are interested in, this is more useful after some rearranging:
$$h^2f_{i,j} = -4u_{i,j} + u_{i-1,j} + u_{i+1,j} + u_{i,j-1} + u_{i,j+1},$$
\begin{equation} \label{eq:poisson}
u_{i,j} = \frac{1}{4}(u_{i-1,j} + u_{i+1,j} + u_{i,j-1} + u_{i,j+1} - h^2f_{i,j}).
\end{equation}

It can be seen from the top equation that this is now a linear algebra problem, requiring a method of solving
$n^2$ dependent linear equations. This means that we can apply one of several standard iterative methods
for solving the problem. This is an excellent application of using a computer to solve numerically a problem
	which is difficult to solve analytically.

\subsection{Performance Considerations}
While operations on a grid of values are parallelizable, there is a limit to
the effectiveness of multi-threading the algorithm. Take, for example, an average
modern processor with 4 cores each with a 256KB cache and a shared cache of 8MB\@. Each core
can only operate on a grid of 256 by 256 32-bit values before exhausting its cache.
A performance drop will occur if the grid size per core exceeds the core's cache
size. For example, should the grid size exceed 1440 by 1440 32-bit values,
the shared 8MB cache will
be filled, resulting in a massive performance penalty. For this reason, na\"{i}vely splitting
the grid into 4 areas and multi-threading the iteration may result in sub-par performance.

There are many other aspects of processor design that come into play when writing a program
such as this. Multithreaded synchronization aside, there are many details of caching that
need to be considered. Of course, optimizing a program is just like any experiment in that
it is absolutely vital to apply normal scientific practices when doing such work. Many make
the mistake of simply writing what they believe to be a more optimized version of a program
without measuring the resulting performance and comparing it to the previous version.

I would expect to see the performance enhanced by the use of multithreading, but insignificantly
due to the large amount of memory accesses made by the program throughout solving a given problem.

