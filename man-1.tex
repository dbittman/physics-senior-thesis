\begin{flushleft}
	SOLVE(1)
	\hfill User Commands \hfill
	SOLVE(1)
\end{flushleft}

\begin{tabbing}
\hspace{30pt}\=\hspace{30pt}\=\kill

\textbf{NAME}\\
\> solve - Solve Discrete Poisson Equation\\
\\
\textbf{SYNOPSIS}\\
	\> \texttt{solve [-V] [-m \textit{method}] [-v \textit{verification-data}] \textit{configuration-file}}\\
	\\
\textbf{DESCRIPTION}\\
\> Solve Poisson Equation given boundary conditions on a discrete grid. Capable of\\
\> solving physics problems that reduce to a boundary value problem. Operates on the\\
\> provided configuration file, producing a 2-D array describing the calculated potential.\\
\\
\> \texttt{\textbf{-V}} \\
\> \> Produce a vector plot, where the vectors are the negative gradient of the potential.\\
\\
	\> \texttt{\textbf{-m} \textit{method}} \\
	\> \> Use method \textit{method} when solving. Current supported values are \texttt{jacobi} for\\
	\> \> Jacobi Iteration, and \texttt{sor} for Successive Over-Relaxation.\\
\\
	\> \texttt{\textbf{-v} \textit{verification-data}} \\
	\> \> Use the data file \textit{verification-data} to compare with the generated potential. The \\
	\> \> format for this file must be that of the Python library pickle serializing a 2-D array.\\
\\
\textbf{CONFIGURATION}\\
\> The configuration file format is specified by a series of commands on lines. A single line\\
\> can contain at most one command. A command must be contained within one line.\\
\> Comments begin with a \#, and comment out the rest of that line. Valid commands are\\
\> as follows, where italics indicates something to be replaced by one token:\\
\\
	\> \texttt{gridsize \textit{size}}\\
	\> \> Set the size of the grid. The grid is always a square, and this command \textbf{must}\\
	\> \> come before any other.\\
	\\
	\> \texttt{cell \textit{coords} initial \textit{value}}\\
	\> \> Specify initial value of a cell inside the grid. Analogous to a point charge.\\
	\\
	\> \texttt{dirichlet \textit{coords} \textit{coords} = \textit{value}}\\
	\> \> Specify a dirichlet boundary condition along the interpolated straight line from the\\
	\> \> first set of coordinates to the second with value \textit{value}.\\
\end{tabbing}
\begin{flushleft}
	statics-solver
	\hfill March 2016 \hfill
	SOLVE(1)
\end{flushleft}
\clearpage
\begin{flushleft}
	SOLVE(1)
	\hfill User Commands \hfill
	SOLVE(1)
\end{flushleft}

\begin{tabbing}
\hspace{30pt}\=\hspace{30pt}\=\kill
	\> \texttt{neumann \textit{coords} \textit{coords} \textit{direction} = \textit{value}}\\
	\> \> Specify a neumann boundary condition along the interpolated straight line from the\\
	\> \> first set of coordinates to the second with value \textit{value} across the boundary of the\\
	\> \> cells specified by \textit{direction}, which may be one of \texttt{left, up, right, down}.\\
	\\
	\> The values of \texttt{\textit{coords, size, direction}} must all be one token, that is,\\
	\> they may not contain any whitespace. The contents of \texttt{\textit{value}} and \texttt{\textit{coords}} are\\
	\> as follows:\\
	\\
	\> \texttt{\textit{value}} is an \texttt{expression}.\\
	\\
	\> \texttt{\textit{coords}} is an \texttt{expression} followed by a comma, followed by an \texttt{expression}.\\
	\\
	An \texttt{expression} is a valid Python expression, with access to the python math library, the\\
	current grid class, and the current position in the grid as specified by \texttt{x} and \texttt{y}. For \\
	example, \texttt{math.sin(x*math.pi/grid.len)} is a valid \texttt{expression}.\\
	\\
	\textbf{AUTHOR}\\
	\> Written by Daniel Bittman (\texttt{danielbittman1@gmail.com}). Please submit any bug\\
	\> reports to this email address.\\
	\\
	\textbf{COPYRIGHT}\\
	\> Copyright\copyright Daniel Bittman. License MIT software license. This is free software, and\\
	\> is provided with NO WARRANTY.\\
	\\

\end{tabbing}
\begin{flushleft}
	statics-solver
	\hfill March 2016 \hfill
	SOLVE(1)
\end{flushleft}

