\section{Program Implementation}
\label{app:des}
The procedure of the program implementation was completed in multiple phases designed to section off
the different aspects of the designing the program into more managable pieces while allowing me to ensure
that each transformation that I made to the program was one such that the program did not lose functionality
or correctness at any stage. Before writing any code, I had a general understanding of the math behind the
algorithm. This did not mean, however, that I was completely sure that the implementation would be correct,
thus necessitating testing. As discussed previously, I designed several test cases for which I knew what the
correct solution was.

%\subsection{Designing for Correctness}
The first iteration of the program was a simplified and not optimal version written entirely
in C in order for me to confirm the correctness of my approach. This version made no attempts to
be fast and had the boundary conditions and grid parameters hard-coded into it. I implemented
Jacobi iteration and SOR, both of which arrived at the correct result for the $\sin(x)$ potential
along a wall.

I next turned my attention to the question of how to interface with the program. Hard-coded boundary conditions
are not desireable, and I quickly found it bothersome to have to modify the code in order to change the conditions
to further test the program. I decided to use a python script to process a configuration file and then call into a
C library which would do the heavy lifting. This is a common approach when dealing with processor intensive simulations
because it allows the speed of C while allowing for the ease of python scripting in order to setup the simulation and
process the results. This necessitated defining a structure in C and a class in python which both contained the same
data in a layout that both C and python could agree upon. This ended up being not too complicated, except for the problem
that python would not allow me to define the memory layouts for the actual grid. To solve this problem, I had the C library
export another symbol which would initialize the data structures contained within the grid structure.

The python code was written to act as a parser for a configuration file and evaluate the result of calling the solver library.
Parsing the configuration file was not complicated, so that code was all written without external libraries. Evalutating the
returned data was done with a combination of the \texttt{numpy} and \texttt{matplotlib} python packages, allowing me to process
and display the results.

One this interface was defined and written, I turned back to the C code to improve the performance. The biggest performance increase
was seen by changing the data layouts to be a contiguous block of memory. Previously while making sure my understanding of the math was
correct, I had simply defined the layout of the cells of the grid to be an array of arrays (since it is 2 dimensional), but not necessarily
contiguous. After changing the allocation sceme, the performance improved dramatically. The second most significant improvement came from
changing the C code from an array-of-structures style data layout to a structure-of-arrays layout, as was discussed earlier. This resulted in
another significant performance boost.

After fixing up the general algorithms, I turned to a basic implementation of multithreaded. This took several iterations to get right. The threads
were each given a section of the array to operate on and they shared a global floating point value in order to determine if the simulation is stable, much like before.
Unlike before, this variables now needed to be atomic in order to avoid loss of data, as discussed previously. This slowed down the simulation
drastically because atomically symchronizing a floating point variable is extremely slow. I then changed the code so that each thread would individually keep
track of the stability of their own section of the grid, and when all of them agreed that the simulation was stable, they would complete. This significantly improved
the multithreaded performance, causing it to beat the single-threaded code.

\subsection{SIMD Implementation}
There were two use of SIMD inside the solver, each for doing one iteration of a given cell under one of two situations: with a Neumann boundary conditon, and
without. In the case of a non-Neumann influenced cell, the cell is calculated as a sum of its nearest neighbors along with its initial value scaled by a
factor of the inverse of the grid size. For successive over-relaxation, the code for an individual cells is shown in Listing~\ref{lst:normal}, and an abridged
SIMD version is shown in Listing~\ref{lst:simd-normal}.

\vspace{5mm}

\begin{minipage}{\linewidth}
\begin{lstlisting}[frame=single,label=lst:normal,caption={Iteration code (simplified) for a single cell. Also notice that this code uses structure-of-arrays data layout.}]
float value = grid->values[x][y+1];
value += grid->values[x][y-1];
value += grid->values[x-1][y];
value += grid->values[x+1][y];
value += params->h * grid->initials[x][y];
value *= params->omega / 4.;
grid->values[x][y]=value+(1.f-params->omega)*grid->values[x][y];
\end{lstlisting}
\end{minipage}

\begin{minipage}{\linewidth}
\begin{lstlisting}[frame=single,label=lst:normal,caption={SIMD version of iteration code for a single cell. The loading of the initial vector is not shown for simplicity.}]
__m256 scal = _mm256_set1_ps(params->prefix2);
__m256 res = _mm256_mul_ps(v1, scal);
grid->values[x][y] = sum8(res);
\end{lstlisting}
\end{minipage}

The function calls beginning with \texttt{\_mm} are the SIMD instructions written in function form for C code. In this listing, \texttt{v1} is a vector that contains
the nearest neighbor values along with a scaled version of \texttt{grid->initials[x][y]} and the cell's previous value. In the second line, each of these elements is
multiplied by \texttt{scal}, much like multiplying a mathematical vector by a scaler value. The elements of the vector are then summed in the \texttt{sum8} function,
and the result of this is stored. The two listings are equivalent. The details of the \texttt{sum8} function and how the initial vector \texttt{v1} is loaded are
omitted here for brevity, but can be found in the source code.

The SIMD code that calculates Neumann boundary conditions is both more complicated and was not used in this thesis, so I will not discuss it in detail, except for one
particularly interesting aspect of it which is a common pattern when writing SIMD code. As discussed previously, it is important to avoid branching in code to improve
performance. The calculation for the Neumann boundary conditions involves testing a condition and throwing away a calculation depending on the result. Instead of
using \texttt{if} statements, we can instead always do all of the calculations, and then at the end select the ones that we want. This is akin to transforming the
code,
\begin{addmargin}[4em]{2em}% 1em left, 2em right
\begin{singlespace}
\texttt{uint32\_t x[2] = \{123, 456\};}\\
\texttt{uint32\_t result[2] = \{0, 0\};}\\
\texttt{if(condition1) result[0] = x[0];}\\
\texttt{if(condition2) result[1] = x[1];}\\
\end{singlespace}
\end{addmargin}
into,
\begin{addmargin}[4em]{2em}% 1em left, 2em right
\begin{singlespace}
\texttt{uint32\_t x[2] = \{123, 456\};}\\
\texttt{uint32\_t result[2] = \{0, 0\};}\\
\texttt{result[0] = x[0] \& mask1;}\\
\texttt{result[1] = x[1] \& mask2;}\\
\end{singlespace}
\end{addmargin}
where \texttt{mask1} and \texttt{mask2} are used to select which values to propegate forward in the calculation. The Intel SIMD instructions have a method to
create such masks easily based on a condition (but they do it in a way that does not involve branching). Because in SIMD doing a single multiplication and doing
four are no different in cost, this takes the approach of avoid branching by doing all of the calculations even though some are optional and then selecting the
ones that are wanted at the end.
In the SIMD code for the Neumann boundary conditions,
I use this technique.
