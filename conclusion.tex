\section{Conclusion}

This program correctly solves certain boundary value problems, enabling simulation
of complex electrostatics situations. It does this in an efficient manner by leveraging
proper programming languages and techniques and utilizing the basics of modern processor
design. There were numerous surprises found in developing this software and looking at
the resulting physics that it described.

\subsection{Correctness}

The most important aspect of this program is that is produces an accurate result, no
matter how fast it runs.


\subsection{Performance}

Since the simulation is correct, I was able to then turn my attention towards improving
the performance without reducing the accuracy of the simulation. Most of this was easy to
do without reducing accuracy, as it was simply transforming one valid implementation of
the code into another and measuring to ensure that my new implementation did in fact
cause an improvement. Most of this was done by improving cache locality and reducing
memory accesses. I have left out a lot of this process for brevity, however an overview
of it and the concepts used can be found in appendix~\ref{app:opt}.

The next significant performance gain attempt that I had made was the use of SIMD instructions.
I was hoping for a much more significant performance boost than I saw, which was rather
miniscule in reality. The only small improvement was not surprising in the single threaded
case because I knew that \texttt{gcc} utilizes SIMD when it can, but I did not expect to see
the manual SIMD instructions degrade the performance in the multi-threaded case. I speculate
that this is caused by my manual SIMD code being less cache friendly, which is okay in the
single threaded case and allowed a small performance improvement, but in the multithreaded case
the threads are all contending for limited cache space, so the additional memory usage slowed
them down.

\subsection{Future Work}

