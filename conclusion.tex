\section{Conclusion}

This program correctly solves certain boundary value problems, enabling simulation
of complex electrostatics situations. It does this in an efficient manner by leveraging
proper programming languages and techniques and utilizing the basics of modern processor
design. There were numerous surprises found in developing this software and looking at
the resulting physics that it described.

\subsection{Correctness}

The most important aspect of this program is that is produces an accurate result, no
matter how fast it runs. I was able to use a problem for which I had an analytical
solution to confirm that the simulation was producing accurate results and found that
the results were of high quality for the single threaded and low number of threads
cases. Additionally, I was able to use the solver program to accurately calculate
that the electric potential from an electric dipole drops off as $1/r^2$, further
confirming the correctness of the program.


\subsection{Performance}

Since the simulation is correct, I was able to then turn my attention towards improving
the performance without reducing the accuracy of the simulation. Most of this was easy to
do without reducing accuracy, as it was simply transforming one valid implementation of
the code into another and measuring to ensure that my new implementation did in fact
cause an improvement. Most of this was done by improving cache locality and reducing
memory accesses. I have left out a lot of this process for brevity, however an overview
of it and the concepts used can be found in appendix~\ref{app:opt}.

The next significant performance gain attempt that I had made was the use of SIMD instructions.
I was hoping for a much more significant performance boost than I saw, which was rather
miniscule in reality. The only small improvement was not surprising in the single threaded
case because I knew that \texttt{gcc} utilizes SIMD when it can, but I did not expect to see
the manual SIMD instructions degrade the performance in the multi-threaded case. I speculate
that this is caused by my manual SIMD code being less cache friendly, which is okay in the
single threaded case and allowed a small performance improvement, but in the multithreaded case
the threads are all contending for limited cache space, so the additional memory usage slowed
them down.

The multithreading actually improved the performance significantly more than I expected it to.
I had predicted that the algorithm would be cache-bound due to the large amount of memory accesses
that it makes. This often results in insignificant performance improvements (or downright performance
degredation) when adding multiple threads to a single threaded program. Of course, the cardinal
rule of any scientific experiment (including attempting to optimize a program) is to measure and
compare results. Clearly the multi-threaded case is significantly better in performance, with a trade-off
in accuracy, which may certainly be acceptible.

\subsection{Future Work}

There is still a significant amount of optimization work that can be done on this program. A careful reading
of Intel optimization manuals would likely provide numerous ideas for how to further improve the code
for the program. The manual SIMD code could also be improved, either by utilizing more up-to-date instruction
sets or by improving the existing algorithm. The multithreaded code could also be improved, by allowing the threads
to communicate in a limited way in order to better synchronize their work to improve the quality of the result without
hopefully degrading the performance too drastically. An additional approach could be to use the multithreaded case for
a few hundred iterations before switching over to the single threaded version in order to smooth out and cleanup the
results. This would have a significant advantage for large simulations, as the multithreaded code could get the result
most of the way before letting the single threaded code produce an accurate result in a shorter amount of time than
if the single threaded code ran from the start.

Another significant way in which to improve the real world performance of the program is in addition to optimizing
the speed at which the program can calculate an iteration, to reduce the number of iterations required to
converge on a solution. Indeed, this is what the use of successive over-relaxation achieves, but there may be ways
to further improve this. One such method could be to improve the initial ``guess'' of zero everywhere by dividing
the grid into a much coarser grid and running a few iterations on that before returning to the full grid and completing
the simulation as before.

\begin{center}\rule{2cm}{0.4pt}\end{center}

