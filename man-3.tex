\begin{flushleft}
	SOLVER(3)
	\hfill Libraries \hfill
	SOLVER(3)
\end{flushleft}

\begin{tabbing}
\hspace{30pt}\=\hspace{30pt}\=\kill

\textbf{NAME}\\
\> solver - Library to solve Discreet Poisson Equation\\
\\
\textbf{SYNOPSIS}\\
	\> \texttt{solver.h}\\
	\> \> \texttt{SOLVE\_METHOD\_JACOBI}\\
	\> \> \texttt{SOLVE\_METHOD\_SOR}\\
	\> \texttt{double \textbf{solve}(struct grid *grid, int method);}\\
	\> \texttt{void \textbf{init\_grid}(struct grid *grid);}\\
	\> \texttt{struct grid \{}\\
	\> \> \texttt{int len;}\\
	\> \> \texttt{int iters;}\\
	\> \> \texttt{float **values;}\\
	\> \> \texttt{float **value\_prevs;}\\
	\> \> \texttt{float **initials;}\\
	\> \> \texttt{uint8\_t **dirichlet\_presents;}\\
	\> \> \texttt{float **dirichlets;}\\
	\> \> \texttt{uint8\_t **neumann\_presents;}\\
	\> \> \texttt{float **neumanns[4];}\\
	\>\texttt{\};}\\
\\
\textbf{DESCRIPTION}\\
\> This library provides fast solving of a 2-D discreet boundary value problem using the\\
	\> Poisson equation. The full process is to define a \texttt{struct grid}, and set its \texttt{len} field. Then\\
	\> call \texttt{init\_grid} on the grid to initialize the 2-D contiguous arrays. After that, call \texttt{solve}\\
	\> and pass it the grid and a method, either \texttt{SOLVE\_METHOD\_JACOBI} or \texttt{SOLVE\_METHOD\_SOR}.\\
	\\
	\> The \texttt{solve} function will return when complete, returning back a value describing how\\
	\> confident it is in its result (values closer to zero are better). The \texttt{iters} field in the grid\\
	\> will have been updated to indicate how many iterations the program took. The \texttt{values}\\
	\> field points to a 2-D array of size \texttt{len} by \texttt{len} containing the solution.\\
	\\
	\textbf{AUTHOR}\\
	\> Written by Daniel Bittman (\texttt{danielbittman1@gmail.com}). Please submit any bug\\
	\> reports to this email address.\\
	\\
	\textbf{COPYRIGHT}\\
	\> Copyright\copyright Daniel Bittman. License MIT software license. This is free software, and\\
	\> is provided with NO WARRANTY.\\

\end{tabbing}
\begin{flushleft}
	statics-solver
	\hfill March 2016 \hfill
	SOLVER(3)
\end{flushleft}

